\PassOptionsToPackage{unicode=true}{hyperref} % options for packages loaded elsewhere
\PassOptionsToPackage{hyphens}{url}
%
\documentclass[]{article}
\usepackage{lmodern}
\usepackage{amssymb,amsmath}
\usepackage{ifxetex,ifluatex}
\usepackage{fixltx2e} % provides \textsubscript
\ifnum 0\ifxetex 1\fi\ifluatex 1\fi=0 % if pdftex
  \usepackage[T1]{fontenc}
  \usepackage[utf8]{inputenc}
  \usepackage{textcomp} % provides euro and other symbols
\else % if luatex or xelatex
  \usepackage{unicode-math}
  \defaultfontfeatures{Ligatures=TeX,Scale=MatchLowercase}
\fi
% use upquote if available, for straight quotes in verbatim environments
\IfFileExists{upquote.sty}{\usepackage{upquote}}{}
% use microtype if available
\IfFileExists{microtype.sty}{%
\usepackage[]{microtype}
\UseMicrotypeSet[protrusion]{basicmath} % disable protrusion for tt fonts
}{}
\IfFileExists{parskip.sty}{%
\usepackage{parskip}
}{% else
\setlength{\parindent}{0pt}
\setlength{\parskip}{6pt plus 2pt minus 1pt}
}
\usepackage{hyperref}
\hypersetup{
            pdftitle={Midterm},
            pdfauthor={Mohammad Shehab},
            pdfborder={0 0 0},
            breaklinks=true}
\urlstyle{same}  % don't use monospace font for urls
\usepackage[margin=1in]{geometry}
\usepackage{color}
\usepackage{fancyvrb}
\newcommand{\VerbBar}{|}
\newcommand{\VERB}{\Verb[commandchars=\\\{\}]}
\DefineVerbatimEnvironment{Highlighting}{Verbatim}{commandchars=\\\{\}}
% Add ',fontsize=\small' for more characters per line
\usepackage{framed}
\definecolor{shadecolor}{RGB}{248,248,248}
\newenvironment{Shaded}{\begin{snugshade}}{\end{snugshade}}
\newcommand{\AlertTok}[1]{\textcolor[rgb]{0.94,0.16,0.16}{#1}}
\newcommand{\AnnotationTok}[1]{\textcolor[rgb]{0.56,0.35,0.01}{\textbf{\textit{#1}}}}
\newcommand{\AttributeTok}[1]{\textcolor[rgb]{0.77,0.63,0.00}{#1}}
\newcommand{\BaseNTok}[1]{\textcolor[rgb]{0.00,0.00,0.81}{#1}}
\newcommand{\BuiltInTok}[1]{#1}
\newcommand{\CharTok}[1]{\textcolor[rgb]{0.31,0.60,0.02}{#1}}
\newcommand{\CommentTok}[1]{\textcolor[rgb]{0.56,0.35,0.01}{\textit{#1}}}
\newcommand{\CommentVarTok}[1]{\textcolor[rgb]{0.56,0.35,0.01}{\textbf{\textit{#1}}}}
\newcommand{\ConstantTok}[1]{\textcolor[rgb]{0.00,0.00,0.00}{#1}}
\newcommand{\ControlFlowTok}[1]{\textcolor[rgb]{0.13,0.29,0.53}{\textbf{#1}}}
\newcommand{\DataTypeTok}[1]{\textcolor[rgb]{0.13,0.29,0.53}{#1}}
\newcommand{\DecValTok}[1]{\textcolor[rgb]{0.00,0.00,0.81}{#1}}
\newcommand{\DocumentationTok}[1]{\textcolor[rgb]{0.56,0.35,0.01}{\textbf{\textit{#1}}}}
\newcommand{\ErrorTok}[1]{\textcolor[rgb]{0.64,0.00,0.00}{\textbf{#1}}}
\newcommand{\ExtensionTok}[1]{#1}
\newcommand{\FloatTok}[1]{\textcolor[rgb]{0.00,0.00,0.81}{#1}}
\newcommand{\FunctionTok}[1]{\textcolor[rgb]{0.00,0.00,0.00}{#1}}
\newcommand{\ImportTok}[1]{#1}
\newcommand{\InformationTok}[1]{\textcolor[rgb]{0.56,0.35,0.01}{\textbf{\textit{#1}}}}
\newcommand{\KeywordTok}[1]{\textcolor[rgb]{0.13,0.29,0.53}{\textbf{#1}}}
\newcommand{\NormalTok}[1]{#1}
\newcommand{\OperatorTok}[1]{\textcolor[rgb]{0.81,0.36,0.00}{\textbf{#1}}}
\newcommand{\OtherTok}[1]{\textcolor[rgb]{0.56,0.35,0.01}{#1}}
\newcommand{\PreprocessorTok}[1]{\textcolor[rgb]{0.56,0.35,0.01}{\textit{#1}}}
\newcommand{\RegionMarkerTok}[1]{#1}
\newcommand{\SpecialCharTok}[1]{\textcolor[rgb]{0.00,0.00,0.00}{#1}}
\newcommand{\SpecialStringTok}[1]{\textcolor[rgb]{0.31,0.60,0.02}{#1}}
\newcommand{\StringTok}[1]{\textcolor[rgb]{0.31,0.60,0.02}{#1}}
\newcommand{\VariableTok}[1]{\textcolor[rgb]{0.00,0.00,0.00}{#1}}
\newcommand{\VerbatimStringTok}[1]{\textcolor[rgb]{0.31,0.60,0.02}{#1}}
\newcommand{\WarningTok}[1]{\textcolor[rgb]{0.56,0.35,0.01}{\textbf{\textit{#1}}}}
\usepackage{graphicx,grffile}
\makeatletter
\def\maxwidth{\ifdim\Gin@nat@width>\linewidth\linewidth\else\Gin@nat@width\fi}
\def\maxheight{\ifdim\Gin@nat@height>\textheight\textheight\else\Gin@nat@height\fi}
\makeatother
% Scale images if necessary, so that they will not overflow the page
% margins by default, and it is still possible to overwrite the defaults
% using explicit options in \includegraphics[width, height, ...]{}
\setkeys{Gin}{width=\maxwidth,height=\maxheight,keepaspectratio}
\setlength{\emergencystretch}{3em}  % prevent overfull lines
\providecommand{\tightlist}{%
  \setlength{\itemsep}{0pt}\setlength{\parskip}{0pt}}
\setcounter{secnumdepth}{0}
% Redefines (sub)paragraphs to behave more like sections
\ifx\paragraph\undefined\else
\let\oldparagraph\paragraph
\renewcommand{\paragraph}[1]{\oldparagraph{#1}\mbox{}}
\fi
\ifx\subparagraph\undefined\else
\let\oldsubparagraph\subparagraph
\renewcommand{\subparagraph}[1]{\oldsubparagraph{#1}\mbox{}}
\fi

% set default figure placement to htbp
\makeatletter
\def\fps@figure{htbp}
\makeatother


\title{Midterm}
\author{Mohammad Shehab}
\date{4/1/2020}

\begin{document}
\maketitle

\hypertarget{introduction}{%
\section{Introduction}\label{introduction}}

On April 15, 1912, the largest passenger liner ever made collided with
an iceberg during her maiden voyage. When the Titanic sank it killed
1502 out of 2224 passengers and crew. This sensational tragedy shocked
the international community and led to better safety regulations for
ships. One of the reasons that the shipwreck resulted in such loss of
life was that there were not enough lifeboats for the passengers and
crew. Although there was some element of luck involved in surviving the
sinking, some groups of people were more likely to survive than others.

The titanic.csv file contains data for 1310 of the real Titanic
passengers. Each row represents one person. The columns describe
different attributes about the person including whether they survived
(S), their age (A), their passenger-class (C), their sex (G) and the
fare they paid (X).

We are going to perform a Exploratory Data Analysis to try to understand
the survivors pattern.How was the classes distributes and does it means
if you are on higher class you had more chances of survival ?

\begin{center}\rule{0.5\linewidth}{0.5pt}\end{center}

\hypertarget{eda-goal}{%
\section{EDA Goal}\label{eda-goal}}

We are going to perform an exploratory data analysis to our titanic
dataset, we will ask some questions about our data set to discover the
variables and values, then we will visualize the distributions perform
some transformation and then the model to come up with conclusion and
analysis.

\begin{center}\rule{0.5\linewidth}{0.5pt}\end{center}

\hypertarget{dependencies}{%
\section{Dependencies}\label{dependencies}}

In order to execute our program we have to have the following
dependencies. I will not be loading the titanic dataset from a csv file
however i have installed a package to do this for me.

\begin{Shaded}
\begin{Highlighting}[]
\CommentTok{#install.packages("tidyverse")}
\CommentTok{#install.packages("rpart.plot") }
\CommentTok{#install.packages("caret")}
\CommentTok{#install.packages("e1071")}
\end{Highlighting}
\end{Shaded}

\begin{Shaded}
\begin{Highlighting}[]
\KeywordTok{library}\NormalTok{(tidyverse)}
\end{Highlighting}
\end{Shaded}

\begin{verbatim}
## -- Attaching packages --------------- tidyverse 1.3.0 --
\end{verbatim}

\begin{verbatim}
## v ggplot2 3.3.0     v purrr   0.3.3
## v tibble  2.1.3     v dplyr   0.8.5
## v tidyr   1.0.2     v stringr 1.4.0
## v readr   1.3.1     v forcats 0.5.0
\end{verbatim}

\begin{verbatim}
## -- Conflicts ------------------ tidyverse_conflicts() --
## x dplyr::filter() masks stats::filter()
## x dplyr::lag()    masks stats::lag()
\end{verbatim}

\begin{Shaded}
\begin{Highlighting}[]
\KeywordTok{library}\NormalTok{(rpart.plot) }
\end{Highlighting}
\end{Shaded}

\begin{verbatim}
## Loading required package: rpart
\end{verbatim}

\begin{Shaded}
\begin{Highlighting}[]
\KeywordTok{library}\NormalTok{(caret)}
\end{Highlighting}
\end{Shaded}

\begin{verbatim}
## Loading required package: lattice
\end{verbatim}

\begin{verbatim}
## 
## Attaching package: 'caret'
\end{verbatim}

\begin{verbatim}
## The following object is masked from 'package:purrr':
## 
##     lift
\end{verbatim}

\#Loading the Titanic data

\begin{Shaded}
\begin{Highlighting}[]
\CommentTok{#clean the workspace for every run}
\KeywordTok{rm}\NormalTok{(}\DataTypeTok{list =} \KeywordTok{ls}\NormalTok{())}
\NormalTok{titanic <-}\StringTok{ }\KeywordTok{read.csv}\NormalTok{(}\StringTok{"titanic.csv"}\NormalTok{, }\DataTypeTok{na.strings=}\StringTok{""}\NormalTok{)}
\KeywordTok{colnames}\NormalTok{(titanic)}
\end{Highlighting}
\end{Shaded}

\begin{verbatim}
##  [1] "Survived"                           "Passenger.Class"                   
##  [3] "Name"                               "Sex"                               
##  [5] "Age"                                "No.of.Siblings.or.Spouses.on.Board"
##  [7] "No.of.Parents.or.Children.on.Board" "Ticket.Number"                     
##  [9] "Passenger.Fare"                     "Cabin"                             
## [11] "Port.of.Embarkation"                "Life.Boat"
\end{verbatim}

\hypertarget{framing-questions}{%
\section{Framing Questions}\label{framing-questions}}

\begin{itemize}
\tightlist
\item
  1 - What is the age distrbution on the ship ?
\item
  2 - How the age distrubuted between sex and age ?
\item
  3 - How did the age and sex affected the survival rate ?
\item
  5 - Does being in a specific class affects your survival rate? Can
  this apply for ages alike ?
\item
  4 - How was the fare distributed?
\end{itemize}

\hypertarget{visualisation-of-distribution}{%
\section{Visualisation of
Distribution}\label{visualisation-of-distribution}}

\begin{itemize}
\tightlist
\item
  we will examin the distribution of the passenger's sex.
\end{itemize}

\begin{Shaded}
\begin{Highlighting}[]
\CommentTok{# if we want to filter `na` we can simply say}
\CommentTok{# titanic<-titanic %>%}
\CommentTok{#  filter(!is.na(Sex))}
\CommentTok{# however, we are not cleaning the data.}

\KeywordTok{ggplot}\NormalTok{(}\DataTypeTok{data =}\NormalTok{ titanic) }\OperatorTok{+}
\StringTok{    }\KeywordTok{geom_bar}\NormalTok{(}\DataTypeTok{mapping =} \KeywordTok{aes}\NormalTok{(}\DataTypeTok{x=}\NormalTok{Sex, }\DataTypeTok{fill=}\NormalTok{Sex), }\DataTypeTok{col =} \StringTok{"black"}\NormalTok{, }\DataTypeTok{alpha =} \FloatTok{0.8}\NormalTok{) }\OperatorTok{+}\StringTok{ }
\StringTok{    }\KeywordTok{labs}\NormalTok{(}\DataTypeTok{x =} \StringTok{"Sex"}\NormalTok{, }\DataTypeTok{y =} \StringTok{"Count"}\NormalTok{, }\DataTypeTok{title =} \StringTok{"Sex Distribution"}\NormalTok{) }\OperatorTok{+}
\StringTok{    }\KeywordTok{theme_classic}\NormalTok{()}
\end{Highlighting}
\end{Shaded}

\includegraphics{midterm_files/figure-latex/unnamed-chunk-4-1.pdf}

Since sex is a categorical variable we use a bar chart. The bar chart
shows that there is 1 \texttt{NA} sex which can indicate that the gender
is missing for the individual, the bar chart displays that the number of
males is greater than females.

We can display the following by using the \texttt{count}. The count
function generates an error when using \texttt{NA} with factors so we
remove the \texttt{NA}

\begin{Shaded}
\begin{Highlighting}[]
\NormalTok{titanic<-titanic }\OperatorTok
\StringTok{ }\KeywordTok{filter}\NormalTok{(}\OperatorTok{!}\KeywordTok{is.na}\NormalTok{(Sex))}
\end{Highlighting}
\end{Shaded}

\begin{Shaded}
\begin{Highlighting}[]
\NormalTok{titanic }\OperatorTok
\StringTok{  }\KeywordTok{count}\NormalTok{(Sex)}
\end{Highlighting}
\end{Shaded}

\begin{verbatim}
## # A tibble: 2 x 2
##   Sex        n
##   <fct>  <int>
## 1 Female   466
## 2 Male     843
\end{verbatim}

To examin the distribution of the \texttt{Age} we will use a histogram
chart.

\begin{Shaded}
\begin{Highlighting}[]
\KeywordTok{ggplot}\NormalTok{(}\DataTypeTok{data =}\NormalTok{ titanic) }\OperatorTok{+}
\StringTok{      }\KeywordTok{geom_histogram}\NormalTok{(}\DataTypeTok{mapping =} \KeywordTok{aes}\NormalTok{(}\DataTypeTok{x =}\NormalTok{ Age), }\DataTypeTok{binwidth =} \FloatTok{0.5}\NormalTok{) }\OperatorTok{+}\StringTok{ }
\StringTok{      }\KeywordTok{labs}\NormalTok{(}\DataTypeTok{x =} \StringTok{"Age"}\NormalTok{, }\DataTypeTok{y =} \StringTok{"Count"}\NormalTok{, }\DataTypeTok{title =} \StringTok{"Age Distribution"}\NormalTok{) }\OperatorTok{+}
\StringTok{        }\KeywordTok{theme_classic}\NormalTok{()}
\end{Highlighting}
\end{Shaded}

\begin{verbatim}
## Warning: Removed 263 rows containing non-finite values (stat_bin).
\end{verbatim}

\includegraphics{midterm_files/figure-latex/unnamed-chunk-7-1.pdf}

The \texttt{age} distribution is plotted above in a histrogram which
shows a \texttt{positive\ skewed} distribution, this can give an
indication on the median if we plot this on a box and whiskers.

\begin{Shaded}
\begin{Highlighting}[]
 \KeywordTok{ggplot}\NormalTok{(}\DataTypeTok{data =}\NormalTok{ titanic, }\DataTypeTok{mapping =} \KeywordTok{aes}\NormalTok{(}\DataTypeTok{x =}\NormalTok{ Age), }\DataTypeTok{horizontal =}\NormalTok{ T, }\DataTypeTok{main=} \StringTok{"Age Distribution of Passengers"}\NormalTok{) }\OperatorTok{+}
\StringTok{   }\KeywordTok{labs}\NormalTok{(}\DataTypeTok{x =} \StringTok{"Age"}\NormalTok{, }\DataTypeTok{title =} \StringTok{"Age Distribution along"}\NormalTok{) }\OperatorTok{+}
\StringTok{      }\KeywordTok{geom_boxplot}\NormalTok{()}
\end{Highlighting}
\end{Shaded}

\begin{verbatim}
## Warning: Removed 263 rows containing non-finite values (stat_boxplot).
\end{verbatim}

\includegraphics{midterm_files/figure-latex/unnamed-chunk-8-1.pdf}

This confirms the following: The lower quartile of the box is 20 which
means 25\% where under the age of 19, and 75\% where older than the age
of almost 25. Although we had some outliers which were above the age of
60+. The mean will not give an accurate indication as it has been
affected by the \texttt{outliers} and the \texttt{NA\textquotesingle{}s}
if we calculate the median the below results confirms that the avarage
age 28 where the majority \texttt{75\%} of the passangers who were on
the titanic.

\begin{Shaded}
\begin{Highlighting}[]
\KeywordTok{summary}\NormalTok{(titanic}\OperatorTok{$}\NormalTok{Age)}
\end{Highlighting}
\end{Shaded}

\begin{verbatim}
##    Min. 1st Qu.  Median    Mean 3rd Qu.    Max.    NA's 
##  0.1667 21.0000 28.0000 29.8811 39.0000 80.0000     263
\end{verbatim}

the following observation if we calculate the number of people with age
respectively and display them in a tabular format.

\begin{Shaded}
\begin{Highlighting}[]
\NormalTok{titanic }\OperatorTok
\StringTok{      }\KeywordTok{count}\NormalTok{(}\KeywordTok{cut_width}\NormalTok{(Age, }\FloatTok{0.5}\NormalTok{))}
\end{Highlighting}
\end{Shaded}

\begin{verbatim}
## Warning: Factor `cut_width(Age, 0.5)` contains implicit NA, consider using
## `forcats::fct_explicit_na`
\end{verbatim}

\begin{verbatim}
## # A tibble: 94 x 2
##    `cut_width(Age, 0.5)`     n
##    <fct>                 <int>
##  1 [-0.25,0.25]              1
##  2 (0.25,0.75]               6
##  3 (0.75,1.25]              15
##  4 (1.75,2.25]              12
##  5 (2.75,3.25]               7
##  6 (3.75,4.25]              10
##  7 (4.75,5.25]               5
##  8 (5.75,6.25]               6
##  9 (6.75,7.25]               4
## 10 (7.75,8.25]               6
## # ... with 84 more rows
\end{verbatim}

It will indicate that the 42 passengers are between the age of
\texttt{{[}23.8\ and\ 24.2{]}}. if we to look at the lower bound of the
box we can have something that looks like this

\begin{Shaded}
\begin{Highlighting}[]
\NormalTok{young<-}\StringTok{ }\NormalTok{titanic }\OperatorTok
\StringTok{    }\KeywordTok{filter}\NormalTok{(Age}\OperatorTok{<}\FloatTok{26.8}\NormalTok{)}

\CommentTok{## then we can plot data}
\KeywordTok{ggplot}\NormalTok{(}\DataTypeTok{data =}\NormalTok{ young) }\OperatorTok{+}
\StringTok{      }\KeywordTok{geom_histogram}\NormalTok{(}\DataTypeTok{mapping =} \KeywordTok{aes}\NormalTok{(}\DataTypeTok{x =}\NormalTok{ Age)) }\OperatorTok{+}\StringTok{ }
\StringTok{      }\KeywordTok{labs}\NormalTok{(}\DataTypeTok{x =} \StringTok{"Age"}\NormalTok{, }\DataTypeTok{y =} \StringTok{"Count"}\NormalTok{, }\DataTypeTok{title =} \StringTok{"Passengers Under 26.8 years of age"}\NormalTok{) }\OperatorTok{+}
\StringTok{        }\KeywordTok{theme_classic}\NormalTok{()}
\end{Highlighting}
\end{Shaded}

\begin{verbatim}
## `stat_bin()` using `bins = 30`. Pick better value with `binwidth`.
\end{verbatim}

\includegraphics{midterm_files/figure-latex/unnamed-chunk-11-1.pdf}

we notice how quickly the observation have changed, now it's showing a
\texttt{negatively\ skewed\ distribution} towards the right.If we use a
histrogram and correlate with the Age/Sex it might give us a better
indication.Hence, we can observe that the Males younger on the ship are
more in numbers with comparison the females.

\begin{Shaded}
\begin{Highlighting}[]
\KeywordTok{ggplot}\NormalTok{(}\DataTypeTok{data =}\NormalTok{ young, }\DataTypeTok{mapping =} \KeywordTok{aes}\NormalTok{(}\DataTypeTok{x =}\NormalTok{ Age, }\DataTypeTok{color =}\NormalTok{ Sex)) }\OperatorTok{+}
\StringTok{      }\KeywordTok{geom_freqpoly}\NormalTok{(}\DataTypeTok{binwidth =} \DecValTok{1}\NormalTok{)}
\end{Highlighting}
\end{Shaded}

\includegraphics{midterm_files/figure-latex/unnamed-chunk-12-1.pdf}

\begin{Shaded}
\begin{Highlighting}[]
        \KeywordTok{labs}\NormalTok{(}\DataTypeTok{x =} \StringTok{"Age"}\NormalTok{, }\DataTypeTok{y =} \StringTok{"Count"}\NormalTok{, }\DataTypeTok{title =} \StringTok{"Passengers Under 26.8 years of age"}\NormalTok{) }\OperatorTok{+}
\StringTok{          }\KeywordTok{theme_classic}\NormalTok{()}
\end{Highlighting}
\end{Shaded}

\begin{verbatim}
## NULL
\end{verbatim}

We now observe how the survival rate was disributed overall between
ages. it almost makes sense as Females had more chances in surviving
than males based on the age distribution along with older females had
more chances of survival females who are below the age of 40 had a 75\%
chance for survival.

\begin{Shaded}
\begin{Highlighting}[]
\KeywordTok{ggplot}\NormalTok{(titanic,}\KeywordTok{aes}\NormalTok{(Survived, Age, }\DataTypeTok{fill =}\NormalTok{ Sex))}\OperatorTok{+}
\StringTok{  }\KeywordTok{geom_boxplot}\NormalTok{()}
\end{Highlighting}
\end{Shaded}

\begin{verbatim}
## Warning: Removed 263 rows containing non-finite values (stat_boxplot).
\end{verbatim}

\includegraphics{midterm_files/figure-latex/unnamed-chunk-13-1.pdf}

We now observe the class with age.

\begin{Shaded}
\begin{Highlighting}[]
\KeywordTok{ggplot}\NormalTok{(}\DataTypeTok{data =}\NormalTok{ titanic , }\KeywordTok{aes}\NormalTok{(}\DataTypeTok{x =} \KeywordTok{as.factor}\NormalTok{(Passenger.Class), }
                            \DataTypeTok{y =}\NormalTok{ Age, }\DataTypeTok{colour =}\NormalTok{ Survived)) }\OperatorTok{+}
\KeywordTok{geom_boxplot}\NormalTok{()}
\end{Highlighting}
\end{Shaded}

\begin{verbatim}
## Warning: Removed 263 rows containing non-finite values (stat_boxplot).
\end{verbatim}

\includegraphics{midterm_files/figure-latex/unnamed-chunk-14-1.pdf}

We observe that people in the first class had a higher chances of
survival however only young people had a higher chances people who were
in the first class and above the age of 38 had 50\% chances of surviving
on the other hand people of where above 60 of age had less than 25\%
chances or surviving.

\begin{itemize}
\tightlist
\item
  How was the fare distributed?
\end{itemize}

\begin{Shaded}
\begin{Highlighting}[]
\KeywordTok{ggplot}\NormalTok{(}\DataTypeTok{data =}\NormalTok{ titanic) }\OperatorTok{+}\StringTok{ }
\StringTok{  }\KeywordTok{geom_point}\NormalTok{(}\DataTypeTok{mapping =} \KeywordTok{aes}\NormalTok{(}\DataTypeTok{x =}\NormalTok{ Passenger.Class, }\DataTypeTok{y =}\NormalTok{ Passenger.Fare), }\DataTypeTok{alpha =} \DecValTok{1} \OperatorTok{/}\StringTok{ }\DecValTok{100}\NormalTok{)}
\end{Highlighting}
\end{Shaded}

\begin{verbatim}
## Warning: Removed 1 rows containing missing values (geom_point).
\end{verbatim}

\includegraphics{midterm_files/figure-latex/unnamed-chunk-15-1.pdf}

Very normal to have fares high based on the class; However we observe
that some people paid very high amount for the same class.

\begin{Shaded}
\begin{Highlighting}[]
\KeywordTok{ggplot}\NormalTok{(}\DataTypeTok{data =}\NormalTok{ titanic , }\KeywordTok{aes}\NormalTok{(}\DataTypeTok{x =} \KeywordTok{as.factor}\NormalTok{(Passenger.Class), }
                            \DataTypeTok{y =}\NormalTok{ Passenger.Fare, }\DataTypeTok{colour =}\NormalTok{ Sex)) }\OperatorTok{+}
\KeywordTok{geom_boxplot}\NormalTok{()}
\end{Highlighting}
\end{Shaded}

\begin{verbatim}
## Warning: Removed 1 rows containing non-finite values (stat_boxplot).
\end{verbatim}

\includegraphics{midterm_files/figure-latex/unnamed-chunk-16-1.pdf}
\texttt{WOW} that was not expected, females with same class as men paid
higher amount for the fare!! some outliers paid 500 for the fare and he
is with the same class as the others.

\hypertarget{refining-and-generating-more-questions}{%
\subsection{Refining and Generating More
Questions}\label{refining-and-generating-more-questions}}

The following exploration leads us to generate more questions.

\begin{itemize}
\tightlist
\item
  Anything is unusual about the age?
\item
  What is the survival rate for elderly?
\item
  Whas was the survival ratio between the sex in the elderlies ?
\end{itemize}

we use the \texttt{coord\_cartesian} to zoom in and see what in unsual
about some points. In the following nothing is unusal about the points
except for 1 passenger who was 80 years of age.

\begin{Shaded}
\begin{Highlighting}[]
\KeywordTok{ggplot}\NormalTok{(titanic) }\OperatorTok{+}\StringTok{ }
\StringTok{  }\KeywordTok{geom_histogram}\NormalTok{(}\DataTypeTok{mapping =} \KeywordTok{aes}\NormalTok{(}\DataTypeTok{x =}\NormalTok{ Age), }\DataTypeTok{binwidth =} \FloatTok{0.5}\NormalTok{) }\OperatorTok{+}
\StringTok{  }\KeywordTok{coord_cartesian}\NormalTok{(}\DataTypeTok{ylim =} \KeywordTok{c}\NormalTok{(}\DecValTok{0}\NormalTok{, }\DecValTok{2}\NormalTok{))}
\end{Highlighting}
\end{Shaded}

\begin{verbatim}
## Warning: Removed 263 rows containing non-finite values (stat_bin).
\end{verbatim}

\includegraphics{midterm_files/figure-latex/unnamed-chunk-17-1.pdf}

\begin{Shaded}
\begin{Highlighting}[]
\NormalTok{unusual <-}\StringTok{ }\NormalTok{titanic }\OperatorTok\StringTok{ }
\StringTok{  }\KeywordTok{filter}\NormalTok{(Age }\OperatorTok{<}\StringTok{ }\DecValTok{1} \OperatorTok{|}\StringTok{ }\NormalTok{Age }\OperatorTok{>}\StringTok{ }\DecValTok{60}\NormalTok{) }\OperatorTok\StringTok{ }
\StringTok{  }\KeywordTok{select}\NormalTok{(Age) }\OperatorTok
\StringTok{  }\KeywordTok{arrange}\NormalTok{(Age)}
\NormalTok{unusual}
\end{Highlighting}
\end{Shaded}

\begin{verbatim}
##        Age
## 1   0.1667
## 2   0.3333
## 3   0.4167
## 4   0.6667
## 5   0.7500
## 6   0.7500
## 7   0.7500
## 8   0.8333
## 9   0.8333
## 10  0.8333
## 11  0.9167
## 12  0.9167
## 13 60.5000
## 14 61.0000
## 15 61.0000
## 16 61.0000
## 17 61.0000
## 18 61.0000
## 19 62.0000
## 20 62.0000
## 21 62.0000
## 22 62.0000
## 23 62.0000
## 24 63.0000
## 25 63.0000
## 26 63.0000
## 27 63.0000
## 28 64.0000
## 29 64.0000
## 30 64.0000
## 31 64.0000
## 32 64.0000
## 33 65.0000
## 34 65.0000
## 35 65.0000
## 36 66.0000
## 37 67.0000
## 38 70.0000
## 39 70.0000
## 40 70.5000
## 41 71.0000
## 42 71.0000
## 43 74.0000
## 44 76.0000
## 45 80.0000
\end{verbatim}

Now we will try to observe the survival rate between the different ages
and sex. but before that we will \texttt{mutate} our data and create an
\texttt{AgeGroup} column which is a calculation of ages.

\begin{Shaded}
\begin{Highlighting}[]
\NormalTok{age_groups <-}\StringTok{ }\NormalTok{titanic }\OperatorTok
\StringTok{  }\KeywordTok{mutate}\NormalTok{(}\DataTypeTok{AgeGroup =} 
           \KeywordTok{ifelse}\NormalTok{(Age }\OperatorTok{<}\StringTok{ }\DecValTok{12}\NormalTok{, }\StringTok{'Child'}\NormalTok{, }
                  \KeywordTok{ifelse}\NormalTok{ ( Age }\OperatorTok{>}\StringTok{ }\DecValTok{12} \OperatorTok{&}\StringTok{ }\NormalTok{Age }\OperatorTok{<}\StringTok{ }\DecValTok{60}\NormalTok{, }\StringTok{'Adult'}\NormalTok{, }
                           \KeywordTok{ifelse}\NormalTok{( Age }\OperatorTok{>}\StringTok{ }\DecValTok{59}\NormalTok{, }\StringTok{'Elderly'}\NormalTok{, }\OtherTok{NA}\NormalTok{)))) }\OperatorTok
\StringTok{    }\KeywordTok{select}\NormalTok{(Survived, Age, AgeGroup, Sex) }\OperatorTok
\StringTok{  }\KeywordTok{arrange}\NormalTok{(Age)}

\CommentTok{## plot it}
\KeywordTok{ggplot}\NormalTok{(}\DataTypeTok{data =}\NormalTok{ age_groups, }\DataTypeTok{mapping =} \KeywordTok{aes}\NormalTok{(}\DataTypeTok{x =}\NormalTok{ Age, }\DataTypeTok{y =}\NormalTok{ AgeGroup, }\DataTypeTok{color=}\NormalTok{Survived)) }\OperatorTok{+}\StringTok{ }
\StringTok{  }\KeywordTok{geom_point}\NormalTok{()}
\end{Highlighting}
\end{Shaded}

\begin{verbatim}
## Warning: Removed 263 rows containing missing values (geom_point).
\end{verbatim}

\includegraphics{midterm_files/figure-latex/unnamed-chunk-18-1.pdf}

This is very new information for me I would have thought that Elderly
are to be saved first (Children, women, and elderly) this is clearly not
the case here. we can plot this on a box and whiskers and see how the
distribution of elderly and correlate it with the class

we can prove this by running the following which shows 28 elderly did
not survive.

\begin{Shaded}
\begin{Highlighting}[]
\NormalTok{survived_elderly <-}\StringTok{ }\NormalTok{age_groups }\OperatorTok
\StringTok{  }\KeywordTok{filter}\NormalTok{(Age }\OperatorTok{>}\StringTok{ }\DecValTok{59}\NormalTok{)  }\OperatorTok
\StringTok{    }\KeywordTok{group_by}\NormalTok{(Survived) }\OperatorTok
\StringTok{    }\KeywordTok{count}\NormalTok{(AgeGroup)}
\end{Highlighting}
\end{Shaded}

\begin{Shaded}
\begin{Highlighting}[]
\NormalTok{age_groups }\OperatorTok
\StringTok{  }\KeywordTok{ggplot}\NormalTok{(}\DataTypeTok{mapping =} \KeywordTok{aes}\NormalTok{(Age)) }\OperatorTok{+}\StringTok{ }
\StringTok{    }\KeywordTok{geom_freqpoly}\NormalTok{(}\DataTypeTok{mapping =} \KeywordTok{aes}\NormalTok{(}\DataTypeTok{colour =}\NormalTok{ Survived), }\DataTypeTok{binwidth =} \DecValTok{1}\OperatorTok{/}\DecValTok{4}\NormalTok{)}
\end{Highlighting}
\end{Shaded}

\begin{verbatim}
## Warning: Removed 263 rows containing non-finite values (stat_bin).
\end{verbatim}

\includegraphics{midterm_files/figure-latex/unnamed-chunk-20-1.pdf}

\hypertarget{conclusion}{%
\subsection{Conclusion:}\label{conclusion}}

This is clearly just the begining of not even scratching the surface of
the \texttt{titanic} dataset I had no idea that the elderly would have a
very chance survival rate and I would have always expected that children
women and elderly clearly that was not the case on the titanic. Fare
distribution was again was a WOW factor for me males paid cheaper amount
for the same class than females!

\end{document}
